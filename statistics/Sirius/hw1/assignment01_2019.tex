\documentclass{article}
\usepackage[utf8]{inputenc}
\usepackage[english, russian]{babel}
\usepackage[margin=0.5in]{geometry}
\usepackage{paralist}
\usepackage{amsthm, amsmath, amsfonts, amssymb}
\usepackage{mathtools} % \mathclap
\usepackage{bm}
\usepackage{dsfont}
\usepackage{hyperref}
\usepackage{tabularx}
\usepackage{graphicx}
\usepackage{multirow}
\usepackage{comment}
\usepackage{xcolor, colortbl}
\usepackage{xifthen, xspace}
\usepackage{caption, subcaption}

\usepackage{sectsty}
\subsectionfont{\normalsize\color{red}}

% The list of general commands
\newcommand{\PI}{3.141592654}
\newcommand{\Sum}{\sum\limits}
\newcommand{\Int}{\int\limits}
\newcommand{\Lim}{\lim\limits}
\newcommand{\Prod}{\prod\limits}
\newcommand{\Intf}{\int\limits_{-\infty}^{+\infty}}
\newcommand{\Sumclap}[1]{\Sum_{\mathclap{#1}}}
\newcommand{\Intclap}[1]{\Int_{\mathclap{#1}}}
\newcommand{\Prodclap}[1]{\Prod_{\mathclap{#1}}}
\newcommand{\Aprod}{\bigodot}
\newcommand{\aprod}{\odot}
\newcommand{\Max}{\max\limits}
\newcommand{\Min}{\min\limits}
\newcommand{\argmax}{\arg\max}
\newcommand{\argmin}{\arg\min}

\newcommand{\RR}{\mathbb{R}}

\newcommand{\lp}{\left(}
\newcommand{\rp}{\right)}
\newcommand{\lf}{\left\{}
\newcommand{\rf}{\right\}}

\newcommand{\Var}{\mathbb{V}}
\newcommand{\Exp}{\mathbb{E}}
\newcommand{\Cov}{\text{Cov}}
\newcommand{\ecdf}{\hat{F}}
\newcommand{\Ecdf}[1]{\hat{F}_n(#1)}

\newcommand{\makebold}[1]{\boldsymbol{#1}}
\newcommand{\angmean}[1]{\left\langle #1 \right\rangle}

\newcommand{\esttheta}{\hat{\theta}}
\newcommand{\estlambda}{\hat{\lambda}}
\newcommand{\estsigma}{\hat{\sigma}}
\newcommand{\estSe}{\hat{\se}}

\newcommand{\estS}{\hat{S}}
\newcommand{\boldX}{\boldsymbol{X}}
\newcommand{\boldY}{\boldsymbol{Y}}
\newcommand{\boldXY}{\boldsymbol{XY}}

\newcommand{\boldx}{\boldsymbol{x}}

\newcommand{\Normal}{\mathcal{N}\xspace}
\newcommand{\LogN}{\mathrm{LogN}\xspace}
\newcommand{\Exponential}{\mathrm{Exp}\xspace}
\newcommand{\hatlambda}{\hat{\lambda}}

\newcommand{\bias}{\ensuremath{\mathrm{bias}\xspace}}
\newcommand{\se}{\ensuremath{\mathrm{se}\xspace}}
\newcommand{\MSE}{\ensuremath{\mathrm{MSE}\xspace}}
\newcommand{\qm}{\ensuremath{\mathrm{qm}\xspace}}



\title{Домашнее задание №1 по курсу \\ <<Математическая Статистика в Машинном Обучении>>}
\author{Школа Анализа Данных}
\date{}

\renewenvironment{itemize}[1]{\begin{compactitem}#1}{\end{compactitem}}
\renewenvironment{enumerate}[1]{\begin{compactenum}#1}{\end{compactenum}}
\renewenvironment{description}[0]{\begin{compactdesc}}{\end{compactdesc}}
% Ненумеруемое замечание
\newtheorem*{note-none}{Замечание}

\begin{document}
\maketitle

\subsection*{Задачи}

\subsubsection*{Задача 1 [1 балла]}
Пусть $\boldX^n = \{X_1, X_2, \dots\}$ --- независимые одинаково распределенные (н.о.р.) случайные величины  с конечными средним $\mu = \Exp(X_1)$ и дисперсией $\sigma^2 = \Var(X_1)$. Покажите, что величины
$$
\angmean{\boldX^n} = \frac{1}{n}\Sum_{i=1}^n X_i, \qquad \estS^2_n = \frac{1}{n - 1}\Sum_{i=1}^n (X_i - \angmean{\boldX^n})^2.
$$
являются \textit{несмещенными} и \textit{состоятельными} оценками среднего $\mu$ и дисперсии $\sigma^2$, т.е. что
\begin{itemize}
	\item $\Exp(\angmean{\boldX^n}) = \mu$ и $\angmean{\boldX^n} \xrightarrow{\Prob} \mu$,
	\item $\Exp(\estS^2_n) = \sigma^2$ и $\estS_n^2 \xrightarrow{\Prob} \sigma^2$.
\end{itemize}

\begin{note-none}
	Конкретно в задачах статистики зачастую под $\boldX^n$ понимается выборка независимых значений случайной величины $X$. В таком случае $\angmean{\boldX^n}$ и $\estS^2_n$ --- оценки среднего и дисперсии по выборке.
\end{note-none}

\subsubsection*{Задача 2 [1 балла]}
Пусть $\boldX^n = \{X_1, X_2, \dots, X_n\}$ и $\boldY^m = \{Y_1, Y_2, \dots, Y_m\}$ --- две выборки н.о.р. случайных величин объема $n$ и $m$, полученных из одного и того же распределения. Пусть $\estS^2_X$ и $\estS^2_Y$ --- несмещенные оценки дисперсий по выборкам $\boldX^n$ и $\boldY^m$ соответственно. Выразите несмещенную оценку дисперсии $\estS_{X, Y}$ суммарной выборки через $\estS^2_X$ и $\estS^2_Y$ и средние $\angmean{\boldX^n}$ и $\angmean{\boldY^m}$.

\subsubsection*{Задача 3 [2 балла]}
Пусть $\boldX^n = \{X_1, \dots, X_n\} \sim \Exponential(\lambda)$, $\hatlambda = 1 / \angmean{\boldX^n}$. Найдите $\bias$, $\se$, $\MSE$ этой оценки. Является ли оценка смещенной? Состоятельной?

\subsubsection*{Задача 4 [2 балла]}
Пусть $\boldX^{n} = \{X_1, \dots, X_n\} \sim \Normal(0, \sigma^2)$. Пусть для оценки параметра $\sigma$ нормального распределения используется выборочное линейное отклонение $\estsigma = \angmean{|\boldX^{n}|} = n^{-1} \sum_{i=1}^{n} |X_i|$. Найдите $\bias$, $\se$, $\MSE$ оценки $\estsigma$. Является ли оценка несмещенной? Если <<нет>>, то постройте исправленную оценку. Найдите $\se$ исправленной оценки. Является ли исправленная оценка $\estsigma$ состоятельной? 

\subsubsection*{Задача 5 [3 балла]}
Пусть $\boldX^{n} = \{X_1,\ldots, X_n\} \sim \Normal(\mu, \sigma^2)$, $\theta = e^{\mu}$ и $\esttheta = e^{\angmean{\boldX_{n}}}$. Найдите аналитически плотность распределения $p_{\esttheta}(x)$ оценки $\esttheta = e^{\angmean{\boldX^{n}}}$, математическое ожидание $\Exp(\esttheta)$,  и дисперсию $\Var(\esttheta)$, а также $\bias$, $\se$, $\MSE$ оценки $\esttheta$. Является ли оценка $\esttheta$ смещенной? Состоятельной?

\subsubsection*{Задача 6 [2 балла]}
Пусть $\Ecdf{x}$ --- эмпирическая функция распределения. Пусть $x$, $y \in \RR$. Найдите ковариацию $\Cov(\Ecdf{x}, \Ecdf{y})$.

\subsubsection*{Задача 7 [2 балла]}
Пусть $\boldX^{n} = \{X_1, \dots, X_n\} \sim F(x)$, и пусть $\Ecdf{x}$ --- эмпирическая функция распределения. Для фиксированных числе $a, b\in \RR$, таких что $a < b$ определим статистический функционал $T(F) = F(b) - F(a)$. Пусть $\esttheta = \Ecdf{b} - \Ecdf{a}$. Найдите оценку $\estSe$ стандартного отклонения и $(1-\alpha)$-доверительный интервал.

\subsubsection*{Задача 8 [2 балла]}
Скачайте данные о качестве красных вин. Постройте график для $\ecdf(x;\boldx^n)$ для уровня кислотности (pH). Для каждой точки $x$ постройте:
\begin{itemize}
	\item 95\%-ый доверительный интервал на основе неравенства Дворецкого-Кифера-Вольфовица.
	\item Асимптотический нормальный 95\%-ый доверительный интервал для значения $F(x)$.
\end{itemize}
По значениям уровня кислотности $\boldx^n$ подсчитайте оценку $T(\boldx^n)$ для функционала $T(F) = F(3.5) - F(3.4)$ и найдите оцените аналитически стандартное отклонение $\estSe$ оценки $T(\boldx^n)$. Постройте асимптотический нормальный 95\%-ый доверительный интервал для $T(F)$.
	
\subsubsection*{Задача 9 [2 балла]}
В процессе очистки питьевой воды выпадает значительный осадок. Для его уменьшения можно воздействовать на разные факторы,
в т.ч. на количество микроорганизмов в жидкости, способствующих окислению органики. В группу из $261$ очистительных установок
был добавлен реагент, подавляющих активность микроорганизмов, а состав остальных $119$ остался без изменений.
Пусть $\theta$ --- разность в средних значениях количества твердых частиц в этих двух группах установок.
Оценить по данным \texttt{WaterTreatment} величину $\theta$, оценить стандартную ошибку оценки, построить
$95\%$ и $99\%$ доверительные интервалы. Какие выводы можно сделать на основе полученных результатов?

\subsubsection*{Задача 10 [2 балла]}
Провести моделирование, чтобы сравнить различные типы доверительных интервалов, построенных с помощью бутстрепа. Пусть $n = 50$, $T(F) = {\int\lp x-\mu\rp^3dF(x)} / \sigma^{3}$ --- коэффициент асимметрии, где $F$ --- логнормальное распределение. Постройте 95\% доверительные интервалы для $T(F)$ (под $F$ понимается распределение элементов выборки $X_1,\ldots,X_n$) по данным $\boldX^n = \{X_1,\ldots,X_n\}$, используя три подхода на основе бутстрепа.

\begin{note-none}
	Выборку из логнормального распределения можно сгенерировать из нормального, сначала сгенерировав выборку н.о.р. величин $\boldY^n = \{Y_1, \ldots,Y_n\} \sim \Normal(0,1)$, после чего положив $X_i = e^{Y_i}$, $i = 1,2,\ldots,n$.
\end{note-none}

\subsubsection*{Задача 11 [2 балла]}
Пусть $\boldX^n = \{X_1,\ldots, X_n\} \sim \Normal(\mu, 1)$, $\theta = e^{\mu}$ и $\esttheta = e^{\angmean{\boldX^n}}$. 
Сгенерируйте выборку $\boldX^n$ из $n = 100$ наблюдений для $\mu = 10$.
Нарисуйте гистограмму  значений $\{\esttheta^*_i\}_{i=1}^B$ бутстрепных оценок. Эта гистограмма является оценкой распределения $p_{\esttheta}(x)$. Сравните ее с настоящим распределением $p_{\esttheta}(x)$.
Используя бутстреп, подсчитайте величину $\se$ и постройте тремя способами $95\%$ доверительный интервал для $\theta$. 

\end{document}